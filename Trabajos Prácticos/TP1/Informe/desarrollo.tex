\section{Desarrollo}

En principio, y con la intencion de entender el problema en profundidad, planteamos un ejemplo que resolvimos manualmente. A partir de este ejemplo
notamos un patron en el comportamiento del sistema. Por lo tanto recrearemos a continuaci\'on dicho ejemplo, para dar una noci\'on clara de nuestra 
evoluci\'on en la comprenci\'on de este problema particular.

Asumamos que recibimos un archivo de entrada con los siguientes parametros:\\
3 3 1 1\\
0.5 2.5 1 500

entonces el problema modelado es de la pinta$^6$:\footnotetext[6]{Aclaramos, para los portadores de una imaginacion poco delirante, 
que la circusferencia negra representa una sanguijuela mutante y el cuadrado celeste, al parabrisas.}

\begin{figure}[H]
    \includegraphics[width=0.3\textwidth]{ejP01}
    \caption{Imagen ilustrativa del parabrisas.}
\end{figure}
Luego lo representaremos como muestra el grafico a continuacion, donde cada intersecion de las lineas representa una variable del modelo, de la cual
debemos averiguar su temperatura.
\begin{figure}[H]
    \includegraphics[width=0.3\textwidth]{ejP02}
    \caption{Representaci\'on visual del modelo discretizado.}
\end{figure}

Notese que, si bien C, D y H pertenecen al radio de una sanguijuela, al formar parte del borde estan acondicionados por el sistema de refrigeracion,
por lo cual su temperatura es $-100$ en lugar de $500$.

Planteando la ecuacion de temperatura para cada variable y despejando, obtenemos el siguiente sistema de ecuaciones $Ax=B$:
\begin{equation}
 a\ =\ -100
\end{equation}
\begin{equation}
 b\ =\ -100
\end{equation}
\begin{equation}
 c\ =\ -100
\end{equation}
\begin{equation}
 d\ =\ -100
\end{equation}
\begin{equation}
 e\ =\ -100
\end{equation}
\begin{equation}
 f - \frac{b}{4} - \frac{e}{4} - \frac{g}{4} - \frac{j}{4}\ =\ 0
\end{equation}
\begin{equation}
 g\ =\ 500
\end{equation}
\begin{equation}
 h\ =\ -100
\end{equation}
\begin{equation}
 i\ =\ -100
\end{equation}
\begin{equation}
 j- \frac{f}{4} - \frac{i}{4} - \frac{k}{4} - \frac{n}{4}\ =\ 0
\end{equation}
\begin{equation}
 k- \frac{g}{4} - \frac{j}{4} - \frac{l}{4} - \frac{o}{4}\ =\ 0
\end{equation}
\begin{equation}
 l\ =\ -100
\end{equation}
\begin{equation}
 m\ =\ -100
\end{equation}
\begin{equation}
 n\ =\ -100
\end{equation}
\begin{equation}
 o\ =\ -100
\end{equation}
\begin{equation}
 p\ =\ -100
\end{equation}
 del cual la matriz asociada $A$ es la siguiente:
%Matriz del sistema
\begin{center}
   \begin{tabular}{| c | c | c | c | c | c | c | c | c |c | c |c | c |c | c | c | c | c | c | c | c |c | c |c | c |c | c |c | c |}
     \hline
       & a  & b & c & d & e & f & g & h & i & j & k & l & m & n & o & p  \\ \hline
   (3) & 1  & 0 & 0 & 0 & 0 & 0 & 0 & 0 & 0 & 0 & 0 & 0 & 0 & 0 & 0 & 0 \\ \hline
   (4) & 0  & 1 & 0 & 0 & 0 & 0 & 0 & 0 & 0 & 0 & 0 & 0 & 0 & 0 & 0 & 0 \\ \hline
   (5) & 0  & 0 & 1 & 0 & 0 & 0 & 0 & 0 & 0 & 0 & 0 & 0 & 0 & 0 & 0 & 0 \\ \hline
   (6) & 0  & 0 & 0 & 1 & 0 & 0 & 0 & 0 & 0 & 0 & 0 & 0 & 0 & 0 & 0 & 0 \\ \hline
   (7) & 0  & 0 & 0 & 0 & 1 & 0 & 0 & 0 & 0 & 0 & 0 & 0 & 0 & 0 & 0 & 0 \\ \hline
   (8) & 0  & $-\frac{1}{4} $ & 0 & 0 & $-\frac{1}{4} $ & 1 & $-\frac{1}{4} $ & 0 & 0 & $-\frac{1}{4} $& 0 & 0 & 0 & 0 & 0 & 0 \\ \hline
   (9) & 0  & 0 & 0 & 0 & 0 & 0 & 1 & 0 & 0 & 0 & 0 & 0 & 0 & 0 & 0 & 0 \\ \hline
   (10) & 0  & 0 & 0 & 0 & 0 & 0 & 0 & 1 & 0 & 0 & 0 & 0 & 0 & 0 & 0 & 0 \\ \hline
   (11) & 0  & 0 & 0 & 0 & 0 & 0 & 0 & 0 & 1 & 0 & 0 & 0 & 0 & 0 & 0 & 0 \\ \hline
   (12) & 0  & 0 & 0 & 0 & 0 & $-\frac{1}{4} $ & 0 & 0 & $-\frac{1}{4} $ & 1 & $-\frac{1}{4} $ & 0 & 0 & $-\frac{1}{4} $ & 0 & 0 \\ \hline
   (13) & 0  & 0 & 0 & 0 & 0 & 0 & $-\frac{1}{4} $ & 0 & 0 & $-\frac{1}{4} $ & 1 & $-\frac{1}{4} $ & 0 & 0 & $-\frac{1}{4} $ & 0 \\ \hline
   (14) & 0  & 0 & 0 & 0 & 0 & 0 & 0 & 0 & 0 & 0 & 0 & 1 & 0 & 0 & 0 & 0 \\ \hline
   (15) & 0  & 0 & 0 & 0 & 0 & 0 & 0 & 0 & 0 & 0 & 0 & 0 & 1 & 0 & 0 & 0 \\ \hline
   (16) & 0  & 0 & 0 & 0 & 0 & 0 & 0 & 0 & 0 & 0 & 0 & 0 & 0 & 1 & 0 & 0 \\ \hline
   (17) & 0  & 0 & 0 & 0 & 0 & 0 & 0 & 0 & 0 & 0 & 0 & 0 & 0 & 0 & 1 & 0 \\ \hline
   (18) & 0  & 0 & 0 & 0 & 0 & 0 & 0 & 0 & 0 & 0 & 0 & 0 & 0 & 0 & 0 & 1 \\ \hline
   \end{tabular}
\end{center}
siendo la ecuacion (3) la resultante de despejar la ecuacion de temperatura para la variable $a$, (4) para la variable $b$ y asi sucesivamente.
Ordenandolo de esta manera, la matriz asociada $A$ es una matriz banda de ancho y altura n, donde n es la cantidad de columnas 
de la matriz de temperaturas. Esto lo deducimos observando la figura 2 y la matriz A,
ya que notamos que para averiguar el valor de la variable $v_{i}$ son necesarios$^7$ los valores de las variables $v_{i-n}$, $v_{i-1}$, 
$v_{i+1}$, $v_{i+n}$.
\footnotetext[7]{Siempre y cuando $v_{i}$ no pertenesca a un borde o este cubierta por una sanguijuela.}

Dejaremos el ejemplo hasta aqui, pues la resolucion del sistema no aporta informaci\'on util. Sin embargo, una vez que hallamos esta estructura 
procedimos a realizar la carga de datos de modo que la matriz asociada a nuestro sistema fuese una matriz banda-n. Cabe destacar, que mientras dos 
miembros del grupo buscaban comprender en profundidad las propiedad del modelo, los otros dos implementaron la estructura para la matriz banda$^8$,
y los algoritmos de Eliminaci\'on Gaussiana y Factorizaci\'on LU. Ahora explicaremos el desarrollo del c\'odigo.
\footnotetext[8]{Por dato del enunciado, en ese momento sabiamos que debiamos obtener una matriz banda a partir del sistema, si bien no sabiamos como
organizar los datos para obtenerla.}

El lenguaje empleado para la implementaci\'on es C++. Si bien la elecci\'on del lenguaje podria considerarse arbitraria, se debio principalmente a la 
cantidad de herramientas y librerias provistas por este lenguaje. 

En primer instancia, creamos una clase llamada BDouble para representar decimales de
precision doble (((?) por que mierda)). Luego creamos la clase Matrix, que representa a las matrices como un vector de vectores de BDouble. 
Ademas guarda la dimension de la matriz, el ancho y alto de la banda. Para mejorar la complejidad espacial, solo guarda por fila aquellos
coeficientes pertenecientes a la banda, o sea, aquellos que no valen necesariamente cero. Esta clase provee las funciones basicas para matriz, como 
consultar su dimencion o sumar multiplos de filas entre si(?), pero ademas, contiene las funciones gaussian\_elimination, LU\_factorization, backward\_substitution y 
forward\_substitution. 

Si bien nuestra primera idea fue crear una clase aparte para estos metodos, y por claridad abstraernos en esta de la 
representacion interna de la matriz, al hacerlo nos topamos con un problema en los header's del codigo. (?) Al no poder descubrir el motivo
de dicho problema, recurrimos a la catedra en busca de consejo. Aun asi, no encontramos el motivo. Es por esto que incluimos la implementacion de
estos algoritmos en la clase Matrix. 

Testeamos estos algoritmos con ejemplos peque\~nos escritos a mano, para verificar que se realizaban las cuentas correctamente. Notese que en 
este momento del desarrollo la carga de datos no estaba implementada, por lo que resultaba imposible correr los test provistos por la catedra.
