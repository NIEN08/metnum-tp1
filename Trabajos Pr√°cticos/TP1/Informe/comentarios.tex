\section{Aclaraciones generales}

\subsection{Sobre las descripciones}

La idea de las descripciones es dar al lector, l\'ease corrector, una comprensi\'on minuciosa de la implementaci\'on de cada filtro. Es por este 
motivo que vamos a plantear un esquema general para las descripciones, a fin de hacer m\'as amena la lectura. 

En principio, daremos una descripci\'on del efecto que debemos conseguir mediante f\'ormulas l\'ogicas y la explicaci\'on de las mismas. Luego presentaremos la idea general de la 
implementaci\'on, seguida por una explicaci\'on detallada del movimiento y la transformaci\'on de los datos mediante el uso de las instrucciones SIMD.

\subsection{Sobre la experimentaci\'on}

La experimentaci\'on aqu\'{i} presentada se realiz\'{o} con las funciones provistas por la c\'{a}tedra, en los laboratorios del DC. Cada item presenta un gr\'{a}fico, 
con una breve explicaci\'{o}n previa, de los datos considerados para su realizaci\'{o}n y del objetivo del experimento que representa. 
 
 Luego de cada gr\'{a}fico se muestran las conclusiones obtenidas del an\'{a}lisis del mismo. 
 
\subsection{Sobre los gr\'{a}ficos}

Las barras de cada gr\'{a}fico representan el promedio de 10 mediciones tomadas, con descarte de 4 outliers a partir del experimento $Secuencial$ $vs.$ $Vectorial$
para el filtro CropFlip, salvo que se especifique lo contrario en la descripci\'on del gr\'afico. Cada barra tiene la indicaci\'{o}n de qu\'{e} representa 
en la leyenda del gr\'{a}fico. La varianza de las mediciones se presenta junto con el promedio, en la parte superior de cada barra.

El lector atento notar\'a que los gr\'aficos que se presentan en este informe no tienen un \'unico formato. Esto se debe al hecho de que no hay un \'unico autor,
sino que es un trabajo 
grupal, y como tal, la personalidad propia de los integrantes influy\'o a la hora de elegir c\'omo representar la informaci\'on. As\'i mismo, consideramos que este 
hecho agrega un toque
personal al trabajo, por lo que decidimos no homogeneizar los gr\'aficos, preservando as\'i el trabajo original de cada integrante. Queda a criterio del lector 
intentar adivinar 
qu\'e formato eligi\'o cada integrante del grupo.

