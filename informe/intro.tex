\section{Introducción}

En este informe se detalla el diseño e implementación de algoritmos que modelan los métodos de Eliminación Gaussiana y Factorización LU,
para resolución de sistemas de ecuaciones lineales. También se presentan algoritmos para la aplicación de la fórmula de Sherman-Morrison y
una estructura para el almacenamiento de matrices banda, con el fin de mejorar la complejidad espacial y temporal. Luego realizamos los experimentos para analizar la calidad de las soluciones, performance de los algoritmos, realismo del modelado, etc.

Esto se realiza al resolver un problema práctico mediante el modelado y discretización del mismo. Este problema consiste en, dado un parabrisas P que está siendo atacado por sanguijuelas mutantes, a las que llamaremos $S_i$, averiguar la temperatura del punto crítico que se halla en el centro de P. El parabrisas P cuenta con un sistema de refrigeración que mantiene el borde a temperatura constante $-100^o$C. Cada sanguijuela $S_i$ cuenta con una posición $(x_i, y_i)$, un radio $r_i$ y una temperatura $t_i$. Si un punto del parabrisas, $P_{k,j}$, pertenece al área afectada por radio $r_i$ de la sanguijuela $S_i$ y no pertenece al borde de P, la temperatura de dicho punto es $t_i$, donde por área afectada nos referimos a la circunferencia de radio $r_i $ y centro $(x_i, y_i)$. Si pertenece a mas de un radio, es decir $P_{k,j} \in {r'_i}, i\in (1..l) $, entonces su temperatura será  $t_{k,j}=max_{i\in (1..l)}(t_i) $. Si el punto $P_{k,j}$ no pertenece a ninguno de los anteriores, su temperatura esta determinada por la siguiente fórmula:

\begin{equation}\label{eq:calor}
\frac{\partial^2T(x,y)}{\partial x^{2}}+\frac{\partial^2 T(x,y)}{\partial y^{2}} = 0.
\end{equation}

Además, si el punto crítico supera los $235^o $C, se desea saber si eliminando tan solo una sanguijuela (¿y que sanguijuela?) se puede lograr que la temperatura se reduzca por debajo de este margen, pues de lo contrario el parabrisas se romperá.

Como P tiene una superficie continua y dado que es imposible representar valores continuos en una PC, debemos discretizar nuestro sistema. Por lo tanto vamos a modelar la superficie de P de tal forma que P sea representado por una matriz de temperaturas. Es decir, solo consideraremos los puntos h-distantes para el modelado, donde h es un parámetro del problema. 
%Por simplicidad, se asume que el ancho y el largo, a y b respectivamente, son multiplos de h.
Solo serán afectados por una sanguijuela $S_i$ los puntos h-distantes que estén incluidos en el radio $r_i$ de $S_i$. Si el radio de una sanguijuela $S_i$ no afecta ningún punto de la discretización, esta no se tendrá en cuenta en el modelado del problema. Una vez discretizado el problema de esta manera, podemos$^2$ estimar la ecuación (1) de la siguiente forma:

\begin{equation}
t_{ij} \ =\ \frac{ t_{i-1,j} + t_{i+1,j} + t_{i,j-1} + t_{i,j+1}}{4}.\label{eq:calordd}
\end{equation}

\footnotetext[2]{ Explicación al respecto en el apéndice A}

%Por lo tanto, obtenemos la siguiente funcion de temperatura global (lo completare si me viene en gana)

A partir de este modelo, despejando las ecuaciones de calor, obtenemos un sistema de ecuaciones lineales. Para la resolución del sistema resultante, vamos a aplicar dos métodos. Estos serán Eliminación Gaussiana y Factorización LU. En el caso en el que debamos decidir que sanguijuela eliminar, aplicaremos en primera instancia la técnica del Algoritmo Simple. Luego aplicaremos la fórmula de Sherman-Morrison, para optimizar. 

Detallando los métodos utilizados, notamos que el método de Eliminación Gaussiana consta de dos pasos: en primer lugar, transforma el sistema original a uno equivalente donde la matriz es triangular superior, con costo $O(n^3)$ donde n es el número de incógnitas. Luego procede a resolver dicho sistema, con costo $O(n^2)$. Por otro lado, el método de Factorización LU descompone el sistema original
$Ax = b$ en un sistema del tipo $LUx = b$, donde $L$ es una matriz triangular inferior y $U$ es una matriz triangular superior, con costo $O(n^3)$. Luego resuelve los sistemas $Ly = b$ y $Ux = y$ para obtener el valor de solución $x$, con costo $O(n^2)$ en cada caso. En términos asintóticos ambos métodos tienen una complejidad $O(n^3)$, sin embargo, la Factorización LU presenta una ventaja por sobre EG. Si el sistema se debe resolver para un b diferente, no hace falta realizar nuevamente la factorización, sino que simplemente basta con resolver los dos sistemas triangulares usando las mismas matrices L y U lo que resulta en un costo cuadrático, en lugar del costo cúbico asociado a la Eliminación Gaussiana. \\
Notar además que para las matrices de este problema se puede aplicar Eliminación Gaussiana sin pivotear, que es lo mismo que decir que, existe Factorización LU.
%\footnotetext[3]{Para una demostración de esta propiedad consulte a su docente amigo de Metodos %Numericos.}
