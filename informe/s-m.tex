Sherman-Morrison nos proponen la siguiente Formula:

$x = (A + uv^{t})^{-1} b = A^{-1}b -  \dfrac{v^{t}A^{-1}b}{1+v^{t}A^{-1}u} A^{-1}u$

Pero sabemos que:

$A^{-1}b = y \iff Ay = b 
A^{-1}u = z \iff Az = u $

Suponiendo que ya resolvimos el sistema A antes de la modificacion, tenemos su descomposicion, $ LU=A $. Por lo que basta resolver: $\\\
Ly_{2} = y $ y $ Lz_{2}=z$

Luego, $ Uy=y_{2} $ y $ Uz=z_{2}$
Una vez calculados $ y $ y $ z $ estamos en condiciones de resolver la expresion. Notemos que

$ \dfrac{v^{t}A^{-1}b}{1+v^{t}A^{-1}u} $ es un escalar.

Finalmente $ x = y +  \dfrac{v^{t}y}{1+v^{t}z} z$


En cuanto a $ uv^{t} $ recordemos que vamos a usar sherman-morrison para sistemas que cambiar una sola fila i, por lo que nos basta con
conseguir que $ uv^{t} $ sea una matriz que modifique apropiadamente dicha fila. Por lo tanto tomamos a $u$ como el i-esimo vector de la base canonica,
y a $v$ como la modificacacion que nesesitamos para tranformar la fila i.








%sola componente/coordenada. Por lo que podemos suponer que la modificacion sera: $ A + \alpha E_{ij} $ donde 
%$ \alpha = \beta - A_{ij} $ con $\beta $ el nuevo valor para la posicion y $ E_{ij} $ la matriz con todos $ 0 $ excepto un $ 1 $ en la posicion $i,j $.
%Ademas sabemos que $ E_{ij} = e_{i} e^{t}_{j} $ donde $ e_{i} $ es el vector como fila con todos ceros excepto un uno en la posicion $ i $ y $ e^{t}_{j} $ es el vector como columna con todos ceros excepto un uno en la posicion $ j $.
%Finalmente $A + uv^{t} =A +\alpha e_{i} e^{t}_{j}$.
%Es decir si cambiamos la posicion $ i, j $ por el valor $ \beta $ tomamos $ uv^{t} = \alpha e_{i} e^{t}_{j}$.
