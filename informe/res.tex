\section{Experimentación y discusión}

Presentaremos en esta sección la experimentación y discusión juntas, para cada experimento. Primero 
daremos una breve descripción de la instancia 
experimental, seguido de lo que esperamos observar, es decir, nuestra hipótesis. A continuación mostraremos los 
resultados obtenidos y nuestro análisis respecto a por qué obtuvimos dichos resultados. 

\subsection{Metodología en la generación de instancias}

Para la creación de instancias de prueba, por una cuestión de practicidad, y para evitar sesgar los experimentos, decidimos utilizar instancias generadas aleatoriamente utilizando numpy. Todo el código de generación de datos, así como de corrida de tests, y las instancias utilizadas para la experimentación, se encuentran en la carpeta experimentos. En términos de las distribuciones utilizadas, en cada uno de los experimentos explicaremos las particularidades de la elección de distribución, pero aquí presentamos una tabla mostrando la distribución utilizada para la generación de los parámetros de las sanguijuelas:

\begin{center}
\begin{tabular}{l | l | l | l | l | l | l}
Experimento & Instancia & \#Sanguijuelas & X & Y & Radio & Temperatura\\ \hline
$1$ & $1$ & $50$ & $\mathcal{U}(45, 55)$ & $\mathcal{U}(45, 55)$ & $|\mathcal{N}(0.5, 10)|$ & $\mathcal{E}(1/100)$\\ \hline
$1$ & $2$ & $50$ & $\mathcal{U}(0, 100)$ & $\mathcal{U}(0, 100)$ & $|\mathcal{N}(2, 100)|$ & $\mathcal{E}(1/100)$\\ \hline
$1$ & $3$ & $50$ & $\mathcal{U}(0, 100)$ & $\mathcal{U}(0, 100)$ & $|\mathcal{N}(0.1, 10)|$ & $\mathcal{E}(1/300)$\\ \hline
$2$ & $1$ & $50$ & $\mathcal{U}(0, 100)$ & $\mathcal{U}(0, 100)$ & $\mathcal{U}(0, 10)$ & $\mathcal{E}(1/300)$\\ \hline
$3$ & $1$ & $31$ & Explicado & Explicado & Explicado & $\mathcal{E}(1/300)$\\ \hline
\end{tabular}
\end{center}

Todas las instancias son de $100x100$. En el caso del experimento 3 tomamos $h = 1$ para simplificar la generación de instancias, mientras que en los otros casos utilizamos

$$h \in \{0.5, 0.8, 1.0, 1.25, 2.0, 2.5, 4.0, 5.0, 6.25, 10.0, 12.5\}$$

Explicaremos más en profundidad estas decisiones en las subsecciones correspondientes. La elección de los valores de $h$ fue una tarea complicada: la restricción de $h | a \wedge h | b$ restringe ampliamente el conjunto de valores; fuera de eso, utilizar valores de granularidad demasiado pequeños aumenta demasiado el costo computacional, como veremos más adelante. Encontramos que intentar correr las instancias con un $h < 0.5$ elevaba demasiado el costo temporal de resolver el problema, por lo que utilizamos este valor como punto de comienzo.

\subsection{Experimento 1: variación de la temperatura en función de la granularidad}
%%aca que esta corriendo? lu o eg? Hay que adjuntar ese dato
Para este experimento, nos planteamos como objetivo mostrar que la matriz final de temperaturas es dependiente de la granularidad elegida para la instancia. Desde el punto de vista algorítmico, notemos que la granularidad impacta en las sanguijuelas que terminan determinando la temperatura inicial en los puntos: si una sanguijuela es lo suficientemente chica y está ubicada de forma tal que su radio no llega a impactar en el punto de la discretización más cercano, no será considerada como parte de la matriz. El experimento que ideamos consiste en crear una instancia donde todas las sanguijuelas estén muy cercanas al punto crítico, con radios lo suficientemente pequeños como para que el efecto que mencionamos antes logre aparecer. Tras un largo proceso de prueba y error, determinamos que las distribuciones mencionadas anteriormente eran las que mejor modelaban la idea intuitiva que estábamos buscando probar. En particular, la elección de una distribución exponencial se debe a que buscamos que las sanguijuelas tengan valores relativamente pequeños para que borrarlas logre mostrarnos un cambio concreto en la temperatura del punto crítico.

Intuitivamente esperábamos que la temperatura en el punto crítico fuera aumentando a medida que achicásemos el $h$: suponíamos que aumentar la granularidad forzosamente iba a resultar en que la mayor temperatura de las sanguijuelas cercanas al centro tomase relevancia en el punto crítico e iba a hacer que la temperatura del mismo subiese, mientras que tener una menor granularidad debía hacer que cada vez más sanguijuelas se perdieran en la discretización y forzaría a que las sanguijuelas mejor posicionadas para los nuevos valores de $h$ dominen la temperatura en el punto crítico. De las instancias que corrimos, creemos que el gráfico \ref{fig:exp11} es ampliamente representativo.

\begin{figure}[h]
    \centering
    \includegraphics[width=0.685\textwidth]{experimento 1-1}
    \caption{Variación de la temperatura en función de la granularidad para la primera instancia}
    \label{fig:exp11}
\end{figure}

Es fácil observar que la temperatura del punto critico se estabiliza a medida que aumenta la granularidad, y tiende a ser más bien caótico a medida que aumentamos el valor de h, es decir, disminuimos la granularidad. A diferencia de lo que esperábamos, la temperatura con mayor granularidad no es la mayor de las obtenidas, encontramos que la granularidad de la discretización es más bien un parámetro de regulación para la precisión de la solución: cuanta mayor granularidad, más difícil es que las sanguijuelas que aparezcan cambien sustancialmente la temperatura del punto crítico, ya que un $h$ más pequeño implica que los radios de las sanguijuelas que quedan por descubrir son lo suficientemente chicos como para afectar pocos puntos, y la ecuación de calor parece "balancear" las temperaturas hacia los valores más frecuentes.

A modo de ayudar a la comprensión del experimento y la instancia contemplada, creamos una visualización del mapa de temperaturas generado por esta instancia particular, variando los valores de $h$:

\begin{figure}[h]
    \centering
    \includegraphics[width=0.685\textwidth]{experimento 1-1}
    \caption{Variación de la temperatura en función de la granularidad para la primera instancia}
    \label{fig:exp11}
\end{figure}

Una vez mostrado el impacto de la granularidad en función de la temperatura, queremos analizar el comportamiento de nuestro algoritmo para casos particulares. En la primer instancia, consideramos en que las sanguijuelas sean peque\`nas y próximas al punto critico. Ahora vamos a experimentar con un caso sustancialmente diferente. 

En la segunda instancia vamos a crear sanguijuelas de un radio grande, en comparación a la primer instancia, alejadas del punto critico. En este caso esperábamos, en principio, que la temperatura se mantuviese constante a medida que variamos la granularidad, pues las sanguijuelas son lo suficientemente grandes para no desaparecer con la discretizaci\'on. Pero dados los resultados del experimento anterior, nos replanteamos el experimento y consideramos esperable cierta oscilación. Esto es por nuestra conclusión anterior, de que la granularidad estabiliza la precisión. 

\begin{figure}[h]
    \includegraphics[width=0.685\textwidth]{experimento 1-2}
    \caption{Variación de la temperatura en función de la granularidad para la segunda instancia}
    \label{fig:exp12}
\end{figure}