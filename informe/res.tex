\section{Experimentaci\'on y discuci\'on}

Presentaremos en esta secci\'on la experimentaci\'on y discuci\'on juntas, separadas por experimentos. Primero 
daremos una breve descripci\'on de la instancia 
experimental, seguido de lo que esperamos observar, es decir, nuestra hipotesis. A continuaci\'on mostraremos los 
resultados obtenidos y nuestro analisis respecto a porque obtuvimos dichos resultados. 
Decidimos darle este formato porque creimos mas declarativo mantener toda la
informaci\'on junta. 

\subsection{Experimento 1: Medici\'on de performance respecto a la variaci\'on de la discretizaci\'on}

\subsubsection{Caso A: \textquestiondown Ser o no ser?}

Este experimento presenta un caso generico de parabrisas, donde por generico nos referimos a que seleccionamos la ubicacion,
radio y temperatura de manera aleatoria entre un grupo de valores predeterminados.
Se corre para distintos tama\~no de discretizaci\'on.
La idea de este experimento
es responder la siguiente pregunta: \textquestiondown Terminar hoy o modelar bien? Con esto nos referimos a si es 
preferible sacrificar tiempo de computo en 
pos de modelar mejor, o si por el contrario es mejor alejarse un poco del modelo real, pero obteniendo resultados 
en un tiempo razonable. Por 
supuesto, es obvio que esta decisi\'on depende en mayor medida del uso que se le quiere dar a los resultados y no 
es nuestra intenci\'on conseguir 
una respuesta generica a esta pregunta, sino analizar la relaci\'on entre estos conceptos logrando asi una mejor 
comprenci\'on del problema. 

Nuestra hipotesis es que para h muy chicos, menores al 5\% de la longitud original, el tiempo de computo aumentara
demaciado, sin obtener ninguna mejora significativa en la modelizacion. Claro que para esta hipotesis estamos 
asumiendo no hallarnos en un caso extremo como, por ejemplo, tener gran cantidad de sanguijuelas microscopicas
con temperaturas similares a la del sol o plut\'on.

\subsubsection{Caso B: Una sanguijuela invoca otra sanguijuela}

En este caso, a diferencia de los demas donde variamos solo el h, fijamos la discretizaci\'on y el tama\~no.
La variable en este experimento 
es la cantidad de sanguijuelas. Planteamos este experimento porque nos parecio interesante ver como se modifica 
la performance de la eliminaci\'on 
Gaussiana y la factorizaci\'on LU, a medida que fijamos mayor cantidad de constantes. Esto es porque 
suponemos que, al tener menos coeficientes la mariz, backward y forward deberian ejecutar mas rapido, 
lo que implica que la ejecucion con mayor cantidad de sanguijuelas deberia ser mas la mas rapida.
Esa es nuestra hipotesis en este experimento.
%%version alocada?

\subsection{Experimento 2: Comparaci\'on de temperaturas en el punto critico}

\subsubsection{Caso A: \textexclamdown Mientras varie la granularidad, sirve!}

Para realizar este experimento, el de comparaci\'on de puntos criticos con tres instancias, usaremos como instancia todos los demas experimentos que varien la granularidad.
Esto hace honor a la famosa frase ``el tiempo es oro'' y como no disponemos mucho de ninguno, 
reutilizaremos las corridas de otros experimentos. En general, esperamos que al achicar el h la 
temperatura del punto critico varie, pero no esperamos que aunmente o disminuya siempre. Al 
aunmentar la granularidad por un lado vamos a incluir en el modelo nuevas sanguijuelas, lo que
potencialmente haria que la temperatura del punto critico, y de los demas puntos, aumente. Sin embargo,
cuando agregamos puntos tambien estamos aunmentando la ``disipacion'' del calor, con lo que 
nos referimos a que los sucesivos calculos de promedio bajan la temperatura a medida que te alejas de 
una sanguijuela. Es por eso que esperamos que las variaciones sean tanto positivas como negativas. Y
remarcamos que una hipotesis concreta de la variacion de la temperatura, debe provenir del analisis de 
una instancia concreta, por lo que no se puede crear una hipotesis generica.

\subsubsection{Caso B: Buscando las sanguijuelas perdidas}

Este experimento fue dise\~nado para modelar un caso que se comporte ``mal'' para un 
h grande. Es decir, modelamos este experimento para que 
las sanguijuelas peque\~nas, que son aqui las de mayor temperatura, esten
distribuidas de modo que no afecten el
modelo para discretizaciones grandes. 
Realizamos este experimento como un caso particular de los decriptos en el A, para mostrar una 
hipotesis concreta. En este caso, como al aunmentar el h van a surgir nuevas sanguijuelas, lo
que esperamos es que la temperatura del punto critico, y en general, aunmente. Esto es porque creemos
que la aparicion de nuevas sanguijuelas es mas significativo que el aunmento de disipacion, cuando este
aunmento es relativamente peque\~no, es decir, del orden del 50-25\%.


\subsection{Experimento 3: Comparaci\'on de performances para backtraking al aplicar Sherman-Morrison}

\subsubsection{Caso A: Sobrepoblaci\'on de mini-sanguijuelas}

Este experimento es un caso extremo para probar el metodo que usa la 
formula de Sherman-Morrison. Todas las 
sanguijuelas si afectan un punto es exclusivamente uno,
en todas las discretizaciones planteadas. La hipotesis es que el metodo 4, que
aplica la Formula de Sherman-Morrison tiene mejor performance temporal.

Como deceamos observar la diferencia de performance entre el 3 y 4 metodo
para los casos de sanguijuelas peque\~nas, decidimos no introducir sanguijuelas que afecten mas de un punto.
Esto es porque las tecnicas de backtraking son, en general, de muy mala performance temporal y disponemos
de un tiempo acotado para finalizar la experimentaci\'on. Por esto, nos restringimos a los casos 
particulares que nos competen, dejando para como proyecto a futuro la experimentaci\'on en casos
generados de manera completamente aleatoria. Esto incluye los experimentos previos.

