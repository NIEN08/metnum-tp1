\section{Experimentaci\'on y discuci\'on}

Presentaremos en esta sección la experimentación y discusión juntas, para cada experimento. Primero 
daremos una breve descripción de la instancia 
experimental, seguido de lo que esperamos observar, es decir, nuestra hipótesis. A continuación mostraremos los 
resultados obtenidos y nuestro análisis respecto a por qué obtuvimos dichos resultados. 

\subsection{Experimento 1: variación de la temperatura en función de la granularidad}

En este experimento queremos comprobar como varia la temperatura del punto critico a medida que variamos la granularidad. Para esto, consideramos tres instancias de prueba.

La primera instancia contiene muchas %%%(siendo muchas algun criterio que no se cual)
sanguijuelas de diámetro peque\~no cerca del punto critico, de temperatura elevada. %%%%%%%%%5(hay que poner todos los rangos, de que se considera pequenio elevadi y blah)
Nuestra hipótesis es que en esta caso, a medida que se aumente la granularidad, vamos a obtener mayor temperatura en el punto critico. Esto es porque como las sanguijuelas son muy peque\~nas, si tomamos una discretizaci\'on muy grande, estas pasarían desapercibidas.


\begin{figure}[h]
    \includegraphics[width=0.3\textwidth]{/experimentos/experimento1-1.jpg}
    \caption{grafiquito}
\end{figure}

Podemos observar en el grafico que la temperatura del punto critico se estabiliza a medida que aumenta la granularidad. Sin embargo, a diferencia de lo que esperabamos, la temperatura con mayor granularidad no es la mayor de las obtenidas. Creemos que esto se debe a que 
cuando aumentamos la granularidad tambien aumentamos el efecto de disipacion, lo que hace que
las temperaturas de todas las sanguijuelas entren en consideracion. Por el contrario, al tener una granularidad muy grande, la mayor parte de esas sanguijelas se perdian en la discretizacion. 

Ademas, observamos una gran cantidad de osilaciones no esperadas. Analizando el grafico y nuestra instancia, deducimos que se debe a que, al variar la granularidad, las sanguijuelas pueden o no aparecer, ya que son de rango muy peque\~no. Es decir, si tomamos un h, para el cual un punto de la discretizacion queda sobre la sanguijuela, esta aparecera en el modelo, pero siendo h muy grande, todas las de alrededor no lo haran. Dado el diametro peque\~no, puede ocurrir que si aumentamos muy poco el valor de h, esta misma sanguijuela no aparece en la discretizacion. Tambien debemos considerar que las sanguijuelas se encuentran en los alrededores inmediatos del punto critico. Ahora parece evidente que el que sea o no considerada una sanguijuela, impacta enormemente en la temperatura del punto critico, si bien en principio no lo creimos asi.