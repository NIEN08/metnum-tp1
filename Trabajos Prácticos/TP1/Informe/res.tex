\section{Experimentaci\'on y discuci\'on}

Presentaremos en esta seccion la experimentacion juntas, separadas por experimentos. Primero presentaremos una breve descripcion de la instancia 
experimental, seguido de lo que esperamos observar, es decir, nuestra hipotesis. A continuacion presentaremos los resultados obtenidos y las
concluciones a las que llegamos. Decidimos darle este formato porque creimos mas declarativo mantener toda la informacion junta. 

\subsection{Experimento 1: Ser o no ser?}

Este experimento presenta un caso generico de parabrisas. Se corre para tres tama\~nos distintos de discretizaci\'on. La idea de este experimento
es responder la siguiente pregunta: ¿Terminar hoy o modelar bien? Con esto nos referimos a si es preferible sacrificar tiempo de computo en 
pos de modelar mejor, o si por el contrario es mejor alejarse un poco del modelo real, pero obteniendo resultados en un tiempo razonable. Por 
supuesto, es obvio que esta decision depende en mayor medida del uso que se le quiere dar a los resultados y no es nuestra intencion conseguir 
una respuesta generica a esta pregunta, sino analizar la relaci\'on entre estos conceptos logrando asi una mejor comprecion del problema.

\subsection{Experimento 2: Buscando las sanguijuelas perdidas}

Este experimento fue dise\~nado para modelar un caso que se comporte ``mal'' para un h grande. Es decir, modelamos este experimento para que 
las sanguijuelas peque\~nas, que son aqui las de mayor temperatura, esten distribuidas de modo que no afecten el modelo para discretizaciones grandes. 
Hacemos esto con la intencion de ver cuanto varia la estimaci\'on de temperaturas al variar el h.

\subsection{Experimento 3: Sobrepoblaci\'on de mini-sanguijuelas}

Este experimento es un caso extremo para probar el metodo que usa la formula de Sherman-Morrison. Todas las ssanguijuelas afectan exclusivamente un punto,
en todas las discretizaciones planteadas. Esperamos observar la diferencia de performance entre el 3 y 4 metodo para los casos de sanguijuelas peque\~nas.

\subsection{Experimento 4: Una sanguijuela invoca una sanguijuela}

En este caso, a diferencia de los anteriores donde variamos solo el h, fijamos la discretizacion y el tama\~no. La variable en este experimento 
es la cantidad de sanguijuelas. Planteamos este experimento porque nos parecio interesante ver como se modifica la performance de la eliminaci\'on 
Gaussiana y la factorizaci\'on LU, a medida que fijamos mayor cantidad de constantes.

\subsection{Experimento 5: Comparaci\'on de temperaturas en el punto critico general}